1 序論
 1.1 はじめに
 1.2 本研究の目的
 1.3 本論文の構成
2 背景と問題点
 2.1 デジタル情報の収集
   情報集取方法は多岐に渡る
   コンテンツサービス業者もデータを収集する必要性
   サービスの一例google adplanner
  2.1.1 ユーザに識別子を付与する手法
  Cookiの概要, 図
  2.1.2 クライアントエージェントを用いた情報収集
  ユーザにアプリケーションをインストールしてもらい,情報を収集する手法
  ネット白書の紹介
  すべてのユーザにアプリをいれるので高コスト
  2.1.3 ネットワークトラフィックの分析
   nebuad,phormの利用するDPIの話
  DPIの概要,図
    データベース化する特徴
    通信に秘密に抵触するかも
 2.2 ユーザプライバシの考慮
  個人情報の保守とサービスはトレードオフ
   企業はプライバシへの配慮している
    プライバシマーク制度,図
    トラストE,図
  電気通信事業者法
    ぷららの事件
 電波法
   日常について言及
 2.3 複数の情報統合によるリスク
    上記のようにプライバシに配慮していても情報を複数組み合わせるとでプライバシの脅威になる.
 2.4 プライバシの脅威となりえる情報
     どのくらい立場によって取れる情報があるのかを確かめる
  2.4.1 第三者のユーザ
    Cookie
    サードパーティflash Cookie,図
    Webサーバのアクセスしたログ
    mailサーバのアクセスしたログ
    無線LAN
     同じチャンネルに接続するとパケットを閲覧できる.
     最近WPA.WEPが危ない→参考文献
    bluetooth

  2.4.2 同じネットワークに存在するユーザ
    スイッチの存在
     すべてのポートで取れるわけではない.
    ブロードキャスト,マルチキャストによる通信内容
    →それらを取得すると情報を
     arp,mdns,smnba
    2.4.3 ネットワークを管者するユーザ
    ペイロードまで見れる
   IPSJの発表資料を添付
   
   2.5 ユーザの情報を考慮した情報技術の要求
   これくらいの情報を取れるが,プライバシの関係からとってはいけない。
   条件を制約する必要性がある.
   2.6 まとめ

3 関連研究
 3.1 Web上での情報収集
 3.2 まとめ

4 仮説
 4.1 ネットワーク管理者と利用情報
 ネットワーク管理者は情報を収集する必要がある.
 ユーザの同意が必要
 条件を制約する必要がある

  4.1.1 前提
  想定するネットワーク環境
  システムを利用するユーザ
  情報を収集するユーザ
  
  4.1.2 パケットのヘッダ情報
  条件制約する必要性
  ユーザに同意を得やすい情報
   パケットのヘッダ情報のみを利用する
 
  4.1.3 設計概要
   CSS2009の内容
  4.1.4 ユーザ識別に用いる情報
   パケットヘッダ情報
  起動時間接続時間
  接続位置
  ユーザが利用するサービス
  OSの種類
   利用ポート番号の傾向
   p0fの利用

   IPアドレスとポート番号の類似性がないという調査
  → 送信トラフィックと受信トラフィック量から情報の推測
   →  サービス名がわからなくても,DNSのTTLから通信しているIPアドレスは固定を利用する


 4.2 同じネットワークにおけるユーザと利用情報
 同じネットワークにいるユーザに自身の個人情報を送信しているのではないか
  様々な情報を送信している.前述した内容
  特に多く流れているmdns,smb,bluetoothを元にネットワーク上で取得。プロファイル
 
 4.2.1 前提
   同ネットワークにおけるユーザの取得出来る情報
   ユーザは管理者権限がない
   ブロードキャスト,マルチキャストがスイッチによって制限されていないネットワーク
   有線無線を問わない
  4.2.2 共有ホスト名
   同ネットワークにユーザが取れる情報について
   ネットワーク上でユーザプロファイル
   ネットワーク共有ホスト名をかたっぱしから取得 
   センシングによって場所情報,生活パターンを追う
   MACアドレスとBluetoothID,ホスト名をバインドしてネットワーク上でトレースできる.
  4.2.3 設計概要

4.3 まとめ
 パケットのヘッダ情報を利用してどこまで個人を取得することが出来るのか
 共有ホスト名とその他の情報をバインドするとプロファイル出来るのか


5 検証
 5.1 パケットのヘッダ情報
 5.1.1 検証手法
  ネットワークにおける鬼ごっこ
 5.1.2 検証環境
   WIDE合宿ネットワーク,図
   期間
   nakajimaネットワークの説明,図
   期間
 5.1.3 検証結果
   特定のプロトコルと接続先によるグルーピング
    ssh
     mail
   サービスが分散されていると同じサービスと認識できない
   送信元と送信先のトラフィック量の比較


5.2 共有ホスト名
 5.2.1 検証手法
  研究室のネットワーク上で無線LANの情報を取得して実際にユーザ情報を収集
 5.2.2 検証環境
   研究室のネットワークの無線LAN,有線LANを利用,nakajimaネットワーク
   期間
 5.2.3 検証結果
  どのくらいユーザを取得できるのか
     ORFのポスター表示
   生活リズム,図
      センシングで場所を検知,図
    MACアドレスとホスト名をバインドしてネットワーク上でのプロファイル

6 考察
 6.1 ユーザプライバシの脅威
  検証結果でわかったことと
 6.2 プライバシ保護
7 結論
 7.1 まとめ
 7.2 今後の展望
