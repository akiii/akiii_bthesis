Abstract of Bachelor's Thesis - Academic Year 2009
~ \\

%\begin{flushright}
%Academic Year 2008
%\end{flushright}

\begin{center}
\begin{Large}
\begin{tabular}{|c|} \hline
Risk Analysis and Countermeasures on User Tracking \\
by Digital Information Surveillance
\\
\hline
\end{tabular}
\end{Large}
\end{center}
~  \\
\indent As computer networks have covered various places and
population globally, users transmit various data in numerous occasions,
both intentionally and unintentionally.  As services that utilize the
network increased, the chance of data transmitted on the network being
accumulated and recorded has reached the significant level.  Those
individual data may not be considered as privacy information.
However, as those control data has increased, it became possible to
combine them and produce a single profile of a certain user.  When the
profiling become possible, the information that weren't considered as
a privacy information then becomes a privacy information.

To ensure that the users' privacy aren't intruded, it is necessary to
determine which information could lead the profiling of the user, and
construct a guideline based on the study.  This thesis clarifies the
types of information that could be accumulated to profile a user, and
how those information could be captured on the computer network.The
method proposed in the thesis classifies collectors into three
categories, and different methods of profiling is stated based on the
characteristics of those categories.  The information used for
capturing a user's profile includes: packet header information,
information used for sharing hosts' computing resources, and device
discovery information for Bluetooth devices.  The threats that could
outcome from the profiling include: revealing users' activity history,
discovering when the users are actively using the network, and
determining actual location of the physical computer that is being a
source of the information.The system for capturing and analyzing those information was developed
to present that they could be a threat against privacy information.  The
result showed that both specifying an individual user and profiling the
user's activities is possible based on the method presented in the
thesis.

Based on the evaluation,we discussed cases of collecting these imformation and impact of users privacy.
Additionally, the guidelines for handling those information is
proposed, to ensure that the users' privacy are protected and secured.


~ \\
Keywords : \\
\underline{1. Network Tracking}, \underline{2.Digital Forensics}, \underline{3.Internet Security}, \underline{4.Network Monitoring}

\begin{flushright}
Keio University, Faculty of Policy Management\\
~ \\
\begin{Large}
Yuki Uehara
\end{Large}
\end{flushright}
